\documentclass[czech, 12pt]{article}
\usepackage[T1]{fontenc}
\usepackage[utf8]{inputenc}
\usepackage{babel}
\usepackage{geometry}
\geometry{verbose,a4paper,tmargin=2cm,bmargin=2cm,lmargin=2cm,rmargin=2cm}
\pagestyle{empty}
\usepackage{booktabs}
\usepackage{draftwatermark}
\usepackage{amsmath}
\DeclareMathOperator{\cotg}{cotg}

\begin{document}
\SetWatermarkScale{4}

\title{Praktický výpočet redukcí terestrických měření v prostorové síti}
\date{\today}
\author{Michal Seidl, Tomáš Kubín\\Fakulta stavební, ČVUT v Praze}

\maketitle

\tableofcontents

\section{Úvod}
Před vyrovnáním prostorové sítě v kartézském souřadnicové systému E3, je nutné 
provést redukci měřených zenitových úhlů a vodorovných směrů. Oprava se provádí
z důvodu sbíhavosti tížnic. Při měření v různých místech sítě je vertikální osa
přistroje urovnávána ve směru tížnice, které nejsou rovnoběžné. Z toho plyne, že
zenitové úhly nejsou zpravidla měřené od směru osy $z$ lokálního sytému a
vodorovné úhly nejsou zpravidla v rovině $xy$. Korekce provádí nápravu tohoto
nesouladu.

\section{Přibližný výpočet}
Výrazy pro přibližný výpočet redukce zenitového úhlu $\Delta z_{ij}$ a směru
$\Delta\psi_{ij}$ (Hradilek) jsou:
\begin{align*}
\Delta z_{ij} &= \varphi_{ci} \cos \Delta\sigma_{ij},\\
\Delta \psi_{ij} &= - \varphi_{ci} \cotg z_{ij} \sin \Delta\sigma_{ij},
\end{align*}
kde $z_{ij}$ je zenitový úhel měřený z bodu $i$ na bod $j$, $\Delta\sigma_{ij}
=\sigma_{ij} - \sigma_{ci}$ je rozdíl směrníků a $\varphi_{ci}$ je geocentrický
úhel normál (ke kouli), které procházejí bodem $i$ a bodem centrálním $c$. 




\end{document}
